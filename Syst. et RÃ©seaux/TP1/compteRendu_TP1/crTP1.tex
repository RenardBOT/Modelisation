\documentclass[a4paper,11pt]{exam}
\usepackage{style/styleCompteRendu}
\usepackage{verbatim}
\usepackage{enumitem,cprotect}
\begin{document}

\Noms{Evan PETIT \\ Valentin VERSTRACTE}
\Titre{TP1 -  Utilisation du Shell}

 \exercice{Le shell}{
 Ouvrir un interpréteur de lignes de commandes (shell). 
    \begin{enumerate}
      \item À quoi correspond le préambule qui précède chacune de vos commandes.
\reponse{\commande{ep298479@zola:/home4/ep298479/
Desktop\$ }
\newline Le préambule indique le nom de l'utilisateur, le serveur et le répertoire courant}
      \item Tester les commandes \texttt{hostname} et \texttt{logname}.
\reponse{\commande{\$ hostname  :  zola
\$ logname  :  Aucun identifiant }
\newline Ces commandes renvoient respectivement le nom du serveur et le nom d'utilisateur sur lequel l'utilisateur s'est connecté.  On suppose que cette dernière ne renvoie rien car nous travaillons sous terminal léger.}
      \item Tester la commande \texttt{whoami}.
\reponse{\commande{\$ whoami  :  ep298479}
\newline whoami retourne le nom de l'utilisateur courant}
      \item Regarder qui est connecté sur la station par la commande \texttt{who}.
      \item Utiliser la commande \texttt{echo \$TERM} pour déterminer le type de terminal utilisé.
\reponse{\commande{\$ echo \$TERM  :  xterm }
\newline Il s'agit de {\bf xterm}}    
      \item Utiliser la commande \texttt{uname -a} pour déterminer votre système d'exploitation.
\reponse{\commande{\$ uname -a  :  Linux zola 4.9.0-8-amd64 1 SMP Debian 4.9.144-3.1...}
 \newline Il s'agit donc de {\bf SMP Debian 4.9.144-3.1}}       
      \item Tester la fonction \texttt{clear}. À quoi sert-elle?
\reponse{Cette instruction permet d'effacer la console}
    \end{enumerate}
 }

%/////////////////////////////////////////////////////////////////////////////////////////////////////////////////////
\exercice{Manuel et aide}{
Vous allez voir comment obtenir des informations sur une commande.
    \begin{enumerate}
      \item Tester les différentes commandes d'aide : \texttt{man}, \texttt{whatis}, \texttt{apropos} et \texttt{info}.
\reponse{
\newline{\bf man [commande] :} Affiche dans le terminal la page du manuel (= man) correspondant à la commande recherchée.
\newline{\bf whatis [commande]:} Affiche dans le terminal une description succincte de la commande recherchée.
\newline{\bf apropos [mot-clé]:} Permet de faire une recherche dans le manuel en fonction de mots-clés
\newline{\bf info [commande]:} Même intérêt que man, affiche le résultat au format Info (qui permet de visualiser de la documentation)
}
      \item Que donne \texttt{>man man}.
\reponse{Ouvre la page du manuel au sujet de la commande {\bf man}}
      \item À quoi sert la fonction ls? Tester différentes options possibles.
\reponse{ls : Affiche à la suite tous les fichiers du répertoire courant
\newline \newline OPTIONS
\newline -l : Affiche les noms des fichiers avec plusieurs informations utiles (notamment  droit en lecture, écriture et d'exécution de l'utilisateur, du groupe, et de l'entièreté des utilisateurs, taille, auteur...)
\newline -i : Affiche le numéro d'index de chaque fichier (id des i-noeuds)
\newline -t : Trie en fonction de la date de modification
} 
    \end{enumerate}
}
 %/////////////////////////////////////////////////////////////////////////////////////////////////////////////////////
\exercice{Shell : raccourcis clavier}{
Donner la fonctionnalité des raccourcis du shell proposés ci-dessous (avec un exemple) :
\reponse{\newline
	Ctrl+a : Déplace le curseur au début de la ligne (debut)		\\
	Ctrl+e : Déplace le curseur à la fin de la ligne (fin)	\\
	Ctrl+d : Supprime le caractère situé après le curseur (suppr.)	\\
	Ctrl+b : Recule le curseur d'un caractère (back)		\\
	Ctrl+f : Avance le curseur d'un caractère (forward)		\\
	Ctrl+l : Saute assez de ligne pour donner l'illusion que la console a été effacée comme {\bf clear} le ferait		\\
	Ctrl+u : Le texte situé avant le curseur	\\
	Ctrl+w : Efface le texte situé entre le curseur et le précédent espace (donc le dernier mot)	\\
	Ctrl+r : Permet d'effectuer une recherche dans l'historique des commandes	
}
}
%/////////////////////////////////////////////////////////////////////////////////////////////////////////////////////
\exercice{Organisation du travail}{
Vous allez maintenant créer dans votre HOME une arborescence qui vous permettra d’organiser votre travail. 
Tous doit bien évidemment être réalisé à partir du shell avec des commandes de base.
  \begin{enumerate}
      \item Créer un dossier SystemesEtReseaux. Entrer dans ce dossier.
 \reponse{\commande{>mkdir SystemesEtReseaux\\>cd SystemesEtReseaux}}
       \item Créer un dossier TP. Entrer dans ce dossier.
 \reponse{\commande{>mkdir TP \&\& cd TP}}    
       \item Créer des dossiers TP1 jusqu’à TP3.    
       \item Télécharger les relatifs aux 3 TPs.
       \item Déplacer ces fichiers dans le dossier correspondant (TP1, TP2 ou TP3).  
  \reponse{\commande{>mv /home1/ep298479/ Téléchargements/TP1.pdf TP1/TP1.pdf
}\newline On fait de même pour les autres fichiers du dossier Téléchargement...}         
       \item Créer un fichier texte \texttt{monNom} contenant votre nom.
   \reponse{\commande{>echo "Evan Petit" > monNom}}         
        \item Créer un lien symbolique pointant vers le fichier \texttt{monNom}.
   \reponse{\commande{>ln -s monNom monNom2}}       
        \item Afficher la taille de \texttt{monNom} et du lien symbolique puis comparer les deux valeurs.
   \reponse{\commande{>stat -c \%g monNom \&\& stat -c \%g monNom2\\11\\6
}\newline Le lien symbolique est bien plus petit en taille } 
         \item Utiliser la commande cat pour afficher le texte pointé par le lien symbolique.
   \reponse{\commande{>cat monNom2}} 
          \item Rajouter au fichier \texttt{monNom} le nom du cours. Puis ré-exécuter la commande précédente. Que se passe-t-il?
   \reponse{ \commande{echo "Systèmes et réseaux - TP1" >> monNom \&\& cat monNom2}
   \newline Le lien symbolique redirige correctement vers le fichier actualisé.
} 
       \item Utilisez la commande ls -al sur les fichiers suivants.
       \begin{itemize}
        \item le fichier commandes.txt
        \item le lien symbolique
        \item le dossier SystemesEtReseaux
        \item le contenu du dossier /dev
        \end{itemize}
	À quoi correspond le premier caractère de chaque ligne ?
	\reponse{Il s'agit du type (d pour directory, - pour file, l pour lien symbolique...}
  \end{enumerate}
} 
%/////////////////////////////////////////////////////////////////////////////////////////////////////////////////////
\exercice{Affichage}
{
	    La commande \texttt{echo} permet d'afficher la valeur des variables sur le shell.
	    \begin{enumerate}
	    \item Tester la commande \texttt{>echo "Bonjour à tous!"} 
	    \item Initialiser une variable locale \texttt{var} à 5 (\texttt{>var = 5})
	    \item Tester les commandes \texttt{>echo var} et \texttt{>echo \$var}
\reponse{\commande{> var=5\\> echo var \&\& >echo \$var\\var\\5}}
	    \item Sur un autre shell tester de nouveau la commande \texttt{>echo \$var}. Que se passe-t'il?
\reponse{...}	    
	    \item Tester plusieurs variables d'environnement (globales). Déterminer votre numéro d’utilisateur (uid) sur le système ainsi que le nom et numéro du groupe auquel vous appartenez (gid).
\reponse{...\\ \commande{>...}}	    
	    \item Initialiser plusieurs variables par la commande \texttt{read} puis afficher les.
\reponse{...}	    
	    \end{enumerate}
 } 
%/////////////////////////////////////////////////////////////////////////////////////////////////////////////////////
\exercice{Droits}
{
  \begin{enumerate}
   \item Créer un répertoire \texttt{rep} contenant un fichier \texttt{fich} contenant la date du jour.
   \item À partir du fichier père de \texttt{rep} :
      \begin{enumerate}
       \item Appliquer la commande \texttt{chmod 600 rep}. Puis tester la commande \texttt{cd rep}.
	      \reponse{Résultats :			\\
		Commentaires : 		}
       \item Appliquer la commande \texttt{chmod 500 rep}. Puis tester les commandes \texttt{cd rep}, \texttt{ls rep} et \texttt{echo 'coucou'>rep/fich2}.
	      \reponse{Résultats :			\\
		Commentaires : 		}	
      \item Appliquer la commande \texttt{chmod 300 rep}. Puis tester les commandes \texttt{cd rep}, \texttt{ls rep} et \texttt{echo 'coucou'>rep/fich2}.
	      \reponse{Résultats :			\\
		Commentaires : 		}
      \end{enumerate}
   \item À partir du fichier \texttt{rep} :
      \begin{enumerate}
       \item Appliquer la commande \texttt{chmod 400 fich}. Puis tester les commandes \texttt{cat fich} et \texttt{echo 'coucou'>>fich}.
	      \reponse{Résultats :			\\
		Commentaires : 		}
       \item Appliquer la commande \texttt{chmod 200 fich}. Puis tester les commandes \texttt{cat fich} et \texttt{echo 'coucou'>>fich}.
	      \reponse{Résultats :			\\
		Commentaires : 		}	
      \end{enumerate}
  \end{enumerate}

}


\end{document}


